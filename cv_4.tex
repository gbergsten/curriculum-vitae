%%%%%%%%%%%%%%%%%%%%%%%%%%%%%%%%%%%%%%%%%
% Medium Length Professional CV
% LaTeX Template
% Version 2.0 (8/5/13)
%
% This template has been downloaded from:
% http://www.LaTeXTemplates.com
%
% Original author:
% Trey Hunner (http://www.treyhunner.com/)
%
% Important note:
% This template requires the resume.cls file to be in the same directory as the
% .tex file. The resume.cls file provides the resume style used for structuring the
% document.
%
%%%%%%%%%%%%%%%%%%%%%%%%%%%%%%%%%%%%%%%%%

%----------------------------------------------------------------------------------------
%	PACKAGES AND OTHER DOCUMENT CONFIGURATIONS
%----------------------------------------------------------------------------------------

\documentclass{resume} % Use the custom resume.cls style
\usepackage[dvipsnames]{xcolor}
\usepackage{fmtcount}
\usepackage[hidelinks]{hyperref}
\usepackage{changepage}

\usepackage{enumitem}
\usepackage{etaremune} % reverse enumerate for paper numbering
\setlist[itemize]{nosep,left=0pt}
\setlist[enumerate]{nosep}
\setlist[description]{nosep}

\usepackage[left=1in, right=1in, top=1in, bottom=1in]{geometry} % margins
\newcommand{\tab}[1]{\hspace{.2667\textwidth}\rlap{#1}}
\newcommand{\itab}[1]{\hspace{0em}\rlap{#1}}
\name{Galen Bergsten}
\address{PhD Student $\mid$ \href{mailto:gbergsten@arizona.edu}{gbergsten@arizona.edu}}
\address{Lunar and Planetary Laboratory, University of Arizona}

%customize entries left, center and right


%%%%%%%%%%%%%%%%% Other Stuff %%%%%%%%%%%%%%%%%%
% \definecolor{PrettyPurple}{rgb}{0.35, 0, 0.35}
% \definecolor{PrettyPurple}{rgb}{0.4375, 0.1875, 0.4375}
\definecolor{PrettyPurple}{rgb}{0,0,0}
%dark background color
\definecolor{bgcol}{RGB}{110,110,110}
%accent color
\definecolor{sectcol}{RGB}{160, 131, 196}
% \definecolor{sectcol}{RGB}{171, 122, 203}
% \definecolor{sectcol}{RGB}{171, 132, 196}

\newcommand{\mystrut}{\rule[-.3\baselineskip]{0pt}{\baselineskip}}

\renewenvironment{rSection}[1]{\mystrut{\textcolor{black}{{\large{\textbf{#1}}}}}
% \textcolor{PrettyPurple}{\MakeUppercase{#1}}
\vspace{-5pt} %{\color{Black} \hrule height 2.5pt}
\begin{list}{}{
\setlength{\leftmargin}{0em}
}
\item[]
}{
\end{list}
}


%%%%%%%%%%%%%%%%%%%%%%%%%%%%%%%%%%%%%%%%%%%%%%
 
\begin{document}
\thispagestyle{empty}

% \noindent\begin{tabular*}{\textwidth}{@{\extracolsep{\fill}}l rr}
% PhD Student & \\
% Lunar and Planetary Laboratory, University of Arizona &  \\
% % 1629 E. University Blvd., Tucson, AZ 85719 \\
% % \href{https://www.lpl.arizona.edu/~gbergsten/}{https://www.lpl.arizona.edu/$\sim$gbergsten/}
% \end{tabular*}

%----------------------------------------------------------------------------------------
%	EDUCATION SECTION
%----------------------------------------------------------------------------------------
\vspace{5pt}\hline
\vspace{0pt}

\begin{rSection}{Education}

{\bf Lunar and Planetary Laboratory, University of Arizona} \hfill {Expected \em 2026} 
\\ PhD in Planetary Sciences, Minor in Astrobiology (Thesis Advisor: Dr. Ilaria Pascucci) \hfill \vspace{-5pt}

{\bf University of Utah} \hfill {\em 2020} 
\\ Honors BS in Physics, Minors in Astronomy (Thesis Advisor: Dr. Gail Zasowski) \hfill
\\ BS in Biology, Minor in Environmental \& Organismal Biology \hfill
%Minor in Linguistics \smallskip \\
%Member of Eta Kappa Nu \\
%Member of Upsilon Pi Epsilon \\
\end{rSection}
\vspace{5pt}\hline
\vspace{0pt}

% %----------------------------------------------------------------------------------------
% %	TECHNICAL STRENGTHS SECTION
% %----------------------------------------------------------------------------------------

% \begin{rSection}{Data Analytics Skills }

% \begin{tabular}{ @{} >{\bfseries}l @{\hspace{6ex}} l }
% Programming Languages &  Python, SQL, Bash, IDL, Fortran, C/C++, MATLAB \\
% Python Packages & Pandas, Matplotlib, Numpy, Scipy, F2py, Psycopg2, \\
%   &BeautifulSoup, Selenium, Spacepy, Davitpy, Jupyter \\
% Software \& Tools & HTML, LaTeX, Excel, Mathematica \\
% \end{tabular}

% \end{rSection}

%----------------------------------------------------------------------------------------
%	WORK EXPERIENCE SECTION
%----------------------------------------------------------------------------------------

\begin{rSection}{Research \& Professional Experience} 

% {\bf Alien Earths Team Member}, NASA’s Nexus for Exoplanet System Science \hfill {\em 2021 - Present} 

{\bf Graduate Research \& Teaching Assistant}, University of Arizona \hfill {\em 2020 - Present} \\
{Demographics of exoplanet systems and their dependence on host star properties; atmospheric evolution of small planets; the frequency of Earth-like habitable planets.}\vspace{-5pt}

{\bf Physics and Astronomy REU}, University of Utah \hfill {\em Summer 2018} \\
{Spectroscopic modeling of stellar populations to constrain cluster chemistry and dynamics.}\vspace{-5pt}

{\bf Undergraduate Research \& Teaching Assistant}, University of Utah \hfill {\em 2017 - 2020} \\
{Characterization of spectroscopic signatures in the interstellar medium associated with massive evolved stars; chemical enrichment via supernova remnant ejecta absorption features.}
\end{rSection}
\vspace{5pt}\hline
\vspace{0pt}


%----------------------------------------------------------------------------------------
%	PUBLICATIONS SECTION
%----------------------------------------------------------------------------------------
% usually hand-modify a AASTeX output from ADS
% (\textbf{Bergsten, G.} \ordinalnum{X} author)

\begin{rSection}{Publications} \itemsep -2pt
\begin{etaremune}[nosep]


\item{Hardegree-Ullman, K. K., Apai, D., \textbf{Bergsten, G.} et al. 2022, submitted: \textit{Bioverse: A Comprehensive Assessment of the Capabilities of Extremely Large Telescopes to Probe Earth-like O2 Levels in Nearby Transiting Habitable Zone Exoplanets}}\vspace{-7pt}

\item{Wanderley, F., Kunha, C., Souto, D. et al. (\textbf{Bergsten, G.} \ordinalnum{13} author) 2023, submitted: \textit{Stellar Characterization and Radius Inflation of Hyades M Dwarf Stars from the APOGEE Survey}}\vspace{-7pt}

\item{\textbf{Bergsten, G.}, Pascucci, I., Mulders, G. D. et al. 2022, AJ, 164, 190: \textit{The Demographics of Kepler's Earths and super-Earths into the Habitable Zone}}\vspace{-7pt}

\item{Fernandes, R. B., Mulders, G. D., Pascucci, I. et al. (\textbf{Bergsten, G.} \ordinalnum{4} author) 2022, AJ, 164, 78: \textit{pterodactyls: A Tool to Uniformly Search and Vet for Young Transiting Planets in TESS Primary Mission Photometry}}\vspace{-7pt}

\item{Koskinen, T. T., Lavvas, P., Huang, C. et al. (\textbf{Bergsten, G.} \ordinalnum{4} author) 2022, ApJ, 929 52K: \textit{Mass loss by atmospheric escape from extremely close-in planets}}\vspace{-7pt}

\item{Ashok, A., Zasowski, G., Seth, A., et al. (\textbf{Bergsten, G.} \ordinalnum{5} author) 2021, AJ, 161, 167. \textit{The APOGEE Library of Infrared SSP Templates (A-LIST): High-resolution Simple Stellar Population Spectral Models in the H Band}}

\end{etaremune}
\end{rSection}
\vspace{5pt}\hline
\vspace{0pt}

%----------------------------------------------------------------------------------------
%	TALKS & POSTERS SECTION
%----------------------------------------------------------------------------------------
\begin{rSection}{Selected Talks and Posters} \itemsep -2pt
\begin{enumerate}

% \item{Caltech/IPAC Seminar (Online)} \hfill {\em March 2023} \\
% \textit{The Occurrence Rate of Earth Analogs with Kepler.}

\item{AAS Meeting \#241 (Contributed Talk; In-Person)} \hfill {\em January 2023} \\
\textit{Demographics of Kepler's Small Planets into the Habitable Zone.}

\item{Jet Propulsion Laboratory Exoplanet Journal Club (Online)} \hfill {\em October 2022} \\
\textit{The Demographics of Kepler's Earths and super-Earths into the Habitable Zone.}

\item{SIG2 Monthly Telecon (Online)} \hfill {\em May 2022} \\
\textit{The Demographics of Kepler's Earths and super-Earths into the Habitable Zone.}

\item{Exoplanets IV (Poster; In-Person)} \hfill {\em May 2022} \\
\textit{The Demographics of Kepler's Earths and super-Earths into the Habitable Zone.}

\item{Origins Seminar Series (Seminar; In-Person)} \hfill {\em May 2022} \\
\textit{The Long \& Short of It: the Population of Earths, from Short Periods to the Habitable Zone.}

% \item{Building a Habitable World (Lecture; In-Person)} \hfill {\em February 2022} \\
% \textit{Exoplanets (Detection and Demographics)}

\item{PLATO Conference 2021 (Contributed Talk; Online)} \hfill {\em October 2021} \\
\textit{Kepler’s Small Planets and their Dependence on Stellar Mass.}

\item{TESS Science Conference 2 (Poster; Online)} \hfill {\em August 2021} \\
\textit{Demographics of Small Kepler Planets and their Dependence on Stellar Mass}

\item{Sagan Workshop (Poster; Online)} \hfill {\em July 2021} \\
\textit{Stellar Mass Dependence in the Abundance of Small Kepler Planets.}

\item{AAS Meeting \#233 (Poster; In-Person)} \hfill {\em January 2019} \\
\textit{An APOGEE-2 Survey of the Stellar Populations in the M31 Group}

\end{enumerate}
\end{rSection}
\vspace{5pt}\hline
\vspace{0pt}


%----------------------------------------------------------------------------------------
%	AWARDS SECTION
%----------------------------------------------------------------------------------------
\begin{rSection}{Awards \& Achievements} \itemsep -2pt
\begin{rSubsection}{Honors}{}{}{}
{Best Graduate Student Talk Award (Lunar and Planetary Laboratory Conference)\hfill {\em 2021}}\\
{BS in Physics and Astronomy (University of Utah), Magna cum Laude with Honors \hfill {\em 2020}}\\
{Undergraduate Research Scholar \hfill {\em 2020}}\\
{Crocker Science House Scholar \hfill {\em 2017}}
\end{rSubsection}
% \vspace{-1pt}
\begin{rSubsection}{Scholarships}{}{}{}
{Thomas J. Parmley Scholarship for Outstanding Students in Physics and Astronomy \hfill {\em 2019}}\\
{Walter W. Wada Endowed Scholarship in Physics and Astronomy \hfill {\em 2018}}\\
{Utah Student Success Scholarship \hfill {\em 2016, 2017}}\\
{University of Utah President’s Scholarship \hfill {\em 2016}}
\end{rSubsection}
\end{rSection}
\vspace{10pt}\hline
\vspace{0pt}


%----------------------------------------------------------------------------------------
%	COMMUNITY SECTION
%----------------------------------------------------------------------------------------
\begin{rSection}{Professional Activities} 
\begin{rSubsection}{Science Committees and Affiliations}{}{}{}
{Science Interest Group 2, \textit{Exoplanet Demographics} \hfill {\em 2022 - Present}}\\
{NASA’s Nexus for Exoplanet System Science Alien Earths Team Member \hfill {\em 2021 - Present}}\\
{Study Analysis Group 22, \textit{Investigating an Exoplanet Target Star Archive} \hfill {\em 2020 - 2021}}\\
{American Astronomical Society \hfill {\em 2018 - Present}}\\
{Society of Physics Students (Vice President), University of Utah Chapter \hfill {\em 2016 - 2020}}
\end{rSubsection} 
\begin{rSubsection}{Teaching Assistantships}{}{}{}
{Building a Habitable World - Instructor: Dr. Mark Marley (LPL) \hfill {\em 2022}}\\
{Introductory Mechanics - Instructor: Mr. Adam Beehler (Utah) \hfill {\em 2019}}\\
{Foundations of Astronomy - Instructor: Dr. Gail Zasowski (Utah) \hfill {\em 2018, 2019}}
\end{rSubsection}
\end{rSection}
\vspace{10pt}\hline
\vspace{0pt}


%----------------------------------------------------------------------------------------
%	MENTORSHIP SECTION
%----------------------------------------------------------------------------------------
% \begin{rSection}{Mentorship} 
% {\bf Colin Boecker-Grieme}, Paradise Valley High School \hfill {\em 2022 - Present} \\
% {Project: \textit{Habitability and Terrestrial Analogs of Europa's Subsurface Ocean}}

% {\bf Abhinav Vatsa}, University of Arizona (Undergraduate) \hfill {\em 2022} \\
% {Project: \textit{Searching for Young Habitable Planets around Low-Mass M Dwarfs with TESS}}

% {\bf Abhinav Vishnuvajhala}, BASIS Phoenix High School \hfill {\em 2022} \\
% {Project: \textit{Indicators of Uninhabitable Worlds with Machine Learning}}
% \end{rSection}

%----------------------------------------------------------------------------------------
%	DEIA SECTION
%----------------------------------------------------------------------------------------
\begin{rSection}{Leadership in Inclusion, Diversity, Equity, \& Accessibility} 
\begin{rSubsection}{Department Leadership}{}{}{}
{Journal Club Coordinator, Lunar and Planetary Laboratory \hfill {\em 2022 - Present}}\\
{DEI Committee, Lunar and Planetary Laboratory \hfill {\em 2022 - Present}}\\
{Department Life Committee, Lunar and Planetary Laboratory \hfill {\em 2022 - Present}}\\
{Graduate Student Colloquium Organizer, Lunar and Planetary Laboratory \hfill {\em 2022 - Present}}\\
{Undergraduate Women in Physics \& Astronomy, University of Utah \hfill {\em 2018 - 2020}}
\end{rSubsection}
\begin{rSubsection}{University Leadership}{}{}{}
{Inclusive Leadership Institute, University of Arizona \hfill {\em 2022 - Present}}\\
{Culturally Inclusive Planetary Engagement Workshop, Planetary ReaCH Program \hfill {\em 2022}}
\end{rSubsection}
\begin{rSubsection}{Outreach}{}{}{}
{The Art of Planetary Science Volunteer \hfill {\em 2020 - Present}}\\
{Tucson Festival of Books - Science City Volunteer \hfill {\em 2023}}\\
{University of Utah Observatory Public Viewing Nights Volunteer \hfill {\em 2017 - 2020}}\\
{Outreach Coordinator for Salt Lake City K-12 Public Schools \hfill {\em 2016 - 2020}}
\end{rSubsection}
\end{rSection}
\vspace{10pt}\hline
\vspace{0pt}

\begin{rSection}{Mentorship} 
{\bf Colin Boecker-Grieme}, Paradise Valley High School \hfill {\em 2022 - 2023} \\
{Project: \textit{Habitability and Terrestrial Analogs of Europa's Subsurface Ocean}}\\
{\bf Abhinav Vatsa}, University of Arizona (Undergraduate) \hfill {\em 2022} \\
{Project: \textit{Searching for Young Habitable Planets around Low-Mass M Dwarfs with TESS}}\\
{\bf Abhinav Vishnuvajhala}, BASIS Phoenix High School \hfill {\em 2022} \\
{Project: \textit{Indicators of Uninhabitable Worlds with Machine Learning}}
\end{rSection}



\end{document}
